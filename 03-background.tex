\chapter{Background}

Storage systems have been studied and documented extensively in the systems literature. One could argue that everything worth doing has been done, or alternatively one could assert that the continuing research is evidence that everyone has failed so far, with the truth probably residing somewhere between these two extremes. Storage is an essential component of most systems, though, and the requirements placed on them are almost as various as the solutions that have been proposed.

In this chapter I survey the most important and widely known storage systems, with an emphasis on those that are relevent to service clusters and similar in design to \envoy. I describe some of the major design tradeoffs that emerge and identify the characteristics of service clusters that make existing systems unsuitable or less than ideal for this emerging environment as I justify the need for another storage system.

\section{Distributed file systems}

The Echo distributed file system \cite{birrell93} was designed to provide a single, global namespace that was accessible anywhere and perfectly coherent. It uses write-behind caching for files and directories with ordering constraints to preserve consistency even in the presence of crashes \cite{mann}, and employs a token-passing scheme to manage cache coherency. The top levels of the namespace are hosted on the DNS, with a link to a specific server. From there, servers host branches of the tree with further links to additional servers. Replication is done at the server level with primary and secondary servers that are synchronized. It died when the OS that hosted it wasn't ported to newer hardware.