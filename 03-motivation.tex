\chapter{Service Clusters}

At 9:30~a.m.\ the New York Stock Exchange opens for trading and the computing demands of many financial services companies are focused on the fluctuations of the markets until the closing bell rings at 4:00~p.m. In the hours until the markets open again, the resources of these companies may turn to intensive jobs---such as simulations---for the company itself, or to handling the processing needs of clients. The applications and data sets involved may be completely unrelated to those used during the daytime hours.

Standard business hours drive many other computing patterns as well. A web site will have the highest traffic load when its target audience is awake and in front of a computer. For workers using personal time, this may mean a traffic surge in the morning, another around lunchtime, and a third during the early evening. For email and other services used by businesses, the traffic curve follows business hours, possibly in many time zones. In contrast, game hosts are busier during leisure hours and see many clients otherwise engaged when corporate offices open their doors.

As internet services become more sophisticated, human clients are not the only ones that drive resource usage. Mobile agents and other automated services are increasingly becoming clients as well as servers. Amazon recently unveiled a service to open their storage platform directly for use by others for web services. Major internet services like eBay and Google offer APIs for third parties to interact with their services using custom software, rather than just their usual web interfaces for human clients.

All of these examples benefit from the simple and rapid deployment and management of software services in a range of environments. Recent years have seen a renaissance of machine virtualization technology, this time in cheap, commodity hardware.

In this chapter I start by defining \emph{flexible commodity computation} and discussing some examples that motivate its adoption. I then discuss existing commodity computing systems and how they succeed at exploiting commodity systems but fail at making general-purpose computation a true commodity. I argue that the combination of virtual machine managers and clusters of commodity hardware offers a potent platform for service deployment, which I call \emph{service clusters}. I further argue that existing storage systems are not well suited for this environment when used as a platform for flexible commodity computation, so a new set of requirements and a storage system to match them are called for.

\section{The flexible commodity computation problem}

Commodity computing describes a range of systems for exploiting the cheap and plentiful computing power available in commodity PCs. Instead of building expensive, specialty servers to tackle complex problems, commercial users and researchers are increasingly harnessing the power of many smaller, cheaper machines to achieve the same end. Because of the massive economies of scale in the PC market, the aggregate power that can be had from a group of cheap PCs is much greater than what the equivalent funds could purchase in more powerful, specialized server hardware.

Efforts to use commodity hardware for large computation tasks are orthogonal to the goal of treating computation power itself as a commodity. While clusters of commodity machines may be part of the underlying implementation of a commodity computation service, using specialized server hardware is also a viable option.

\emph{Utility computing}, also called \emph{on-demand computing}, describes the business model of providing computing services and charging based on use of resources, which may include resources reserved but not actually utilized. This may apply to specialized services such as databases or web hosting platforms, or to any service where usage is metered and clients pay based on what they use rather than the capabilities that are available.

\emph{Flexible commodity computation} is to a form of utility computing that allows standard commodity operating systems and tools as well as customized services to be deployed quickly and cheaply, where the basic environment provides commodity operating systems and tools rather than a specialized platform. Deployment costs are proportional to how far a client deviates from a standard, commodity environment, e.g., a standard Linux distribution, rather than the total space required for a service to operate. By providing standard tools and standard environments, one flexible commodity computation service can be swapped for another without significant barriers such as deployment costs, making computation itself a fungible commodity that can be used for a wide range of tasks, large and small.

\subsection{XenoServers}
\subsection{Morgan Stanley}
\subsection{Mobile agents}

\section{Commodity computing}

\subsection{The GRID}
\subsubsection{Volunteer projects}
\subsubsection{Commercial and research projects}
\subsubsection{Planetlab}

\subsection{Commodity clusters}
\subsubsection{Google}
\subsubsection{Beowulf clusters}
\subsubsection{Supercomputer 500 list}

\subsection{Web services}
\subsubsection{Thin clients}
\subsubsection{Mobile clients}
\subsubsection{Hosted services}
email, payroll, tax prep

\subsection{Clusters as a commodity computing platform}

Clusters of commodity hardware offer several benefits over highly provisioned servers or cooperative but dispersed workstations. They offer a degree of centralization that retains many of the benefits of single servers while adding the robustness and scalability of distributed groups of machines. The onus for maintaining a coherent platform is transfered from the hardware vendor in the case of large servers to the software, but the flexibility achieved in return makes the tradeoff profitible.

\subsection{When one is better than many}

Centralized administration and high-speed communication (via shared memory and IPC) are two advantages of traditional servers. Clusters typically put components in a single location with high-speed local networking and a secure physical location. The machines are all owned and administered by a single organization and can have whatever cooling systems and redundant power supplies are appropriate for the application. Systems designed to prevent or avoid any centralized control usually do so for privacy or legal reasons, neither of which is compelling in the service cluster environment. Indeed, some degree of central administration is an essential feature of service clusters. The owner of the cluster needs to control access and admission to the service pool as well as monitor services that are running to ensure that they do not exceed their alloted resources. Given physical proximity and central control, central administration adds convenience without introducing unnecessary penalties in flexibility.

To further emulate the desirable characteristics of centralized servers, clusters must ensure global access to data in the storage system from any constituent node. This is best achieved through a single, global name space with a completely coherent view of all files in the cluster. Trading coherence for performance represents a failure of global access, as concurrent services effectively create private views of the data, requiring a seperate mechanism for restoring the consistency that the file system violated \cite{birrell93}.

Services running on a single server all fail together if the server fails. Correlated failure is generally considered a problem in distributed systems, but it can also simplify the requirements of failure recovery at least in the common case. When a single server fails, the file system is generally restored to a consistent state before its services are restarted and allowed to recover. If the file system fails while the service continues, this is often considered a similarly catastrophic failure of the system, no better than the two parts failing together. In a virtual machine environmenet, the distributed file system manager can be located on the same physical machine as the service that depends on it, restoring this hardware correlation between the two components. Note that the virtual machine monitor already represents a single point of failure for an entire physical machine; adding the file system manager as a second critical software component that cannot be recovered without a restart of the node only enlarges the set of critical tools, it does not create it. The failure of a client service does not enjoy the same priviledged status.

\subsection{When many is better than one}

Scalability is a significant benefit of clusters over centralized servers. Hardware can be added incrementally according to need, which eliminates the need for an accurate forecast of the lifetime demands of the system. In addition, clusters of machines can achieve much greater overall scale than even the largest single machines. Scalability over time and absolute scalability are also features of non-clustered distributed systems, but those lack the high-speed interconnects possible in managed environments (note that I consider a cluster to be defined more by the coherent administrative environment and high-speed networking than by physical proximity; machines in a single cluster could be located in different buildings or at different sites if the other conditions are met).

The use of commodity hardware also allows rapid machine acquisition and cheap prices. Incremental scaling means that newly added hardware can always take advantage of the best price/performance ratios and benefit from the constant downward pressure on prices in a competitive market \cite{fox}. Commodity disks are relatively unreliable, but they are also large and cheap and offer a good source for storage capacity \cite{patterson,warfield} that comes standard with most machines.

The independence of nodes in a cluster offers redundancy that can be exploited for both reliability and availability. This is necessary not only to exceed the expectations of a single system, but to match it as well. Having many parts that can fail independently offers a much higher probability that at least one will fail than that any single component will fail, and a cluster that does not tolerate some component failures will quickly become unusable \cite{birrell93}.

Clusters have mainly been confined to solving ``embarrassingly parallel''\footnote{See \url{http://en.wikipedia.org/wiki/Embarrassingly_parallel}} problems such as search engines, scientific computing, data mining, and other parallel computing.

\section{Service clusters}

Combining groups of services into clusters yields numerous benefits over deployment in isolation. Load balancing can extend not only to multiple participants in a single logical service, but across dispirate services hosted in the same cluster.

\subsection{Services as a management unit}

One of the enduring goals of systems research is to provide a good platform for running applications. Even early systems such as Multics were explicitly intended as infrastructure for higher-level computing services, seeking ``continuous operation in a utility-like manner, with flexible remote access,'' with requirements such as ``convenience of manipulating data and program files, ease of controlling processes during execution and above all, protection of private files and isolation of independent processes'' \cite{corbato}. Forty years later, we have made progress but still recognize similar goals.

\subsection{Hosting services}

\subsection{Virtual machines}

Hardware virtualization is one of the more potent tools available for achieving these ends. Virtual memory systems are a standard tool for isolating processes managed by operating systems, and virtualizing entire hardware platforms extends the ability to partition resources to the operating system level, which is a more flexible unit of management \cite{hand}. Once confined mainly to specialized server hardware, machine virtualization is emerging as an important general purpose management tool. The Xen hypervisor \cite{barham} was the first to make virtualization efficient for commodity hardware, especially for I/O intensive applications.

\subsection{Commodity everything}

\subsection{Heterogeneous workloads}

Many web hosting services routinely oversubscribe their hardware capacity, a practice that allows them to provision based on expected average use rather than maximum potential use. Since clusters can be expanded incrementally, observed average use can become the metric for established providers, especially at large scale. It is prudent to hold some reserve capacity to handle bursts in usage, but even burstiness becomes more predictible at large scales. The more diverse the services using a cluster, the less likely an external event will trigger a burst in usage that overwhelms the capacity of the entire cluster. Designing for overflow capability using machines outside the cluster provides insurance against unexpected traffic \cite{fox}, but using a general-purpose service platform with a heterogeneous load allows averaging to reduce the probability of needing resources beyond those in normal use.

Besides accounting for random bursts of traffic, periodic fluctuactions in demand can be exploited to make better use of resources. Predictable events like business hours, holidays, academic calendars, and stable cycles can significantly influence some service workloads. Planning for these by pairing complementary services in a dynamically-configured cluster can help avoid idle resources and reduce expenses. Systems used heavily during business hours could share with systems used by game servers, assuming that the latter are used more during leisure hours than on company time.

\section{Storage requirements}

\subsection{Distribution}

Three basic architectures are available. A centralized server model is the simplest and satisfies the mobility requirement, but it doesn't scale well and suffers from a single point of failure. This model has been extended quite successfully using RAIDs and SANs on the back end to increase performance and robustness, and by caching on the client end to reduce load, but it is still limited. Ultimately, all decisions are moderated by a single host which must communicate directly with all clients, so at larger scale the network bandwidth becomes a problem as well.

At the other extreme, coordinations is pushed completely to the clients. This is essentially the centralized model turned upside down. Instead of many clients talking to a single server, a single client must communicate with many servers. Instead of one congestion point, every client becomes overwhelmed with bookkeeping. In practice, some structure can be introduced so that the system isn't a full $n$-way clique---the host for each object of storage may become the mediator for access to that object, for example---but making every participant a symmetric peer is mostly useful for privacy or other non-technical concerns.

A more practical approach is to emulate a centralized architecture as much as possible, since this provides the simplest model with the least administrative overhead, and carve up the problem enough to achieve the desired scaling properties. The two extremes offer the greatest conceptual purity, but the practical benefits are to be had somewhere between them. Most P2P services now introduce some kind of a hierarchy that designates some hosts as ``superpeers'' or ``supernodes'' that act as aggregation points for a manageable set of participants. These distinguished hosts only need to communicate directly with other similarly endowed hosts, potentially reducing the visible size of the network by orders of magnitude. The world is small enough and computers big enough that a few orders of magnitude is enough reduction to make many distributed problems tractable.

Another way of viewing this problem is to consider how many manually administered systems are often implemented today. NFS \cite{callaghan} is still one of the most widely-used systems, and a workstation environment may include a series of NFS servers to provide storage for a larger number of client workstations. Each server will be backed by a RAID \cite{patterson} system for improved performance and reliability, and the administrators may divide the file system namespace in order to balance average demand over the different servers. While most file access in uncontended, having a single server that serves all interested clients makes coherence easy (ignoring the effects of caching).

An interesting thought exercise is to consider the same set of workstations locking into a long-term steady-state behavior, and then asking the systems administrator to hand configure the system based on those usage patterns. One reasonable possibility would be to locate files or branches of the namespace tree used exclusively by a single workstation on that workstation. When two or more users require the same file or set of files, it could be managed by the workstation that uses it the most and exported via NFS (or a similar client-server file system protocol) to the other participants. Since contention is relatively rare, this would reduce network traffic for the storage system to a low level for as long as the hypothetical steady-state continues. One of the goals of the Envoy design is to capture this approximate topology.

In service clusters, physical hosts are divided into roughly 10s of services, which may be transient, untrusted, and unreliable. This increases the number of participants requiring access to the storage system by the same factor and exposes difficult security problems. One of the principle benefits of isolating services in individual virtual machines is in reducing the impact of services that fail, either benignly or maliciously, and trusting the administration of the file system to the services negates this benefit.

The environment includes a trusted management domain on each server, however, which introduces a natural aggregation point and a way to sidestep many of those problems.

\subsection{Local impact}

Envoy is designed according to the principle of \textit{local impact}, meaning that the resources consumed directly or indirectly by a service should be as close to that service as possible. If not in the same VM, then on the same machine, or on another machine that has some specific reason to yield its resources to a remote service.

By extension, this principle leads to a topology that is shaped according to runtime conflicts. When there is no reason to suspect contention, machines will prefer to assume complete control over the storage in active use by their client services. If two machines must explicitly coordinate their access to storage, they are treading on overlapping or neighboring storage and implicitly declaring that a conflict is likely to occur.

Previous studies of file system traffic have repeatedly concluded that runtime contention is rare, so Envoy is designed to assume that exclusive access dominates and react to conflicts as they occur rather than optimizing for the occasions when access overlaps.

\subsection{Supporting services}

VM migration \cite{clark} is essential for load balancing and uninterrupted service in the presence of hardware servicing.

To support migration, a file system cannot be tied to a single physical machine. Any given file system image must be as mobile as the service that relies on it.

One of the attractive features of running individual services in virtual machines is the flexibility of management this model offers. Services can be decoupled, instances can be created and destroyed easily in response to need, and a new service can be deployed without dedicating a new machine to it, removing and old service to make room, or studying the potential interactions that would occur if a shared server was the only option. To this end, lightweight operations are required to instantiate and clone services.

The unit of management is an entire operating system on a virtual machine. Installing an operating system for each service instance would be prohibitively slow, labour intensive, and wasteful of space. A typical installation of Linux using Fedora Core~3 takes about 50 terabytes of space, and an automated installation is unlikely to give the right set of installed packages and running daemons, so some customization will be necessary.

Instead of trying to automate installations, Envoy offers lightweight operations to fork a file system.

\section{Summary}
