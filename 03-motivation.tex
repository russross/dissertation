\chapter{Service Clusters}

The standardization of the shipping container revolutionized the logistics industry, which in turn has had a significant impact on global commerce over the past 50 years. Building a box hardly seems like the stuff of revolutions, but the cheap availability of foreign goods and easy access to global markets that characterizes modern economies owns much to exactly that. The significance of the container is illustrated partly by what it offers, namely, flexibility and efficiency, but equally significant is what it does not offer. It is not an engine, a vehicle, a route, a company, a service, or any of the things necessary to transport goods from one place to another. Instead, it is a neutral, common ground. Those wishing to ship something can pack according to well-known dimensions using widely-available tools, and those offering transport services can use trains, ships, trucks, or whatever method of transport and whatever routing system allows them to offer competitive service while making a profit.

The computer industry is in need of a shipping container. While the computer hardware industry is increasingly viewed as a commodity business, computation as a commodity service is still in its infancy. The components are all there: PCs are powerful, networks are fast, disks are big, operating systems are flexible and efficient, and everything is cheap. The Top~500 supercomputer list is dominated by clusters make from commodity hardware\footnote{See \url{http://www.top500.org/}}, and companies such as Google have used commodity hardware to solve large commercial problems instead of relying on powerful, special-purpose hardware that is more powerful and more reliable, but also much more expensive. These efforts have been highly successful, but they still revolve around using commodity hardware to build platforms that are customized for a particular task or class of tasks. Like oil tankers or passenger trains, they are efficient and well-suited to their intended markets, but not easily adapted to other kinds of clients.

Containers have succeeded for several reasons. They are generic enough to support an enormous range of cargo, but rigid enough to pack tightly together and stack neatly. They can be pulled individually behind trucks, strung together on trains, or packed onto huge cargo ships for ocean transport. They can be moved and loaded easily and quickly using crains, and they can move from one ship to the next without any changes. Clients can fill as many containers as they need and only pay according to what they use.

Many of these same characteristics would help to make commodity computation a reality. Clients should be able to deploy a wide range of computation and communication services, but vendors should also be able to manage them in a generic way. Clients should be able to use their own commodity machines and software to implement services, but have them run equally well in a large, commercial setting. Likewise, deployment costs should be low and procedures simple, and redeployment costs should not create onerous barriers to changing vendors. Finally, small services using very few resources and those spanning many dedicated machines should be able to coexist without interfering with each other and with clients paying according to what they use.

In this chapter I argue that a platform of \emph{service clusters} approximates this ideal, using commodity hardware and commodity operating systems and tools, isolated and managed using virtual machine monitors. The \emph{service} is the unit of management, allowing arbitrary tasks to run, each isolated in a virtual machine. While I do not prescribe a particular operating system or set of tools, I do suggest that a small set of common, commodity platforms can be agreed on, and that clients and vendors can both benefit from adhering to them. Clients can then package their custom services as everything that differs from a standard platform and send it to the vendor that can host it according to their needs.

Not every task is best served by a common platform---oil tankers will continue to excel at transporting oil---but many jobs that today run on custom installations could be better implemented as commodity computation tasks that are cheaper for clients to run and profitable for vendors to host. I discuss other efforts toward commodity computation and how my proposal overlaps or relates to them.

Finally, I do not propose a complete solution. Instead, I argue for the essential characterstics that distinguish service clusters from other large-scale uses of commodity computers, while leaving much of the general infrastructure to others. I focus on the storage needs of the platform and identify how a suitably designed file system can use the commodity hardware in a cluster to cheaply and easily support deployment and management of services built on commodity tools.

\section{Commodity computation}

Services are harder to package as commodities than goods. The quality of oil, the purity of precious metals, the strength of steel, and the composition of building materials can all be objectively measured. Compatibility with standards, quality of workmanship, energy consumption, and feature sets make consumer electronic devices comparable, and even food can be graded and compared, at least at the level of basic ingredients. Services can also be commodities, but only when competing offerings can be compared on price and quality, and when customers can move freely between providers without prohibitive lock-in effects.

This section starts by defining computation as a service amenable to commoditization, then explores how this proposal relates to existing platforms and projects with similar goals.

\subsection{The flexible commodity computation problem}

Commodity computing describes a range of systems for exploiting the cheap and plentiful computing power available in commodity PCs. Instead of building expensive, specialty servers to tackle complex problems, commercial users and researchers are increasingly harnessing the power of many smaller, cheaper machines to achieve the same end. Because of the massive economies of scale in the PC market, the aggregate power that can be had from a group of cheap PCs is much greater than what the equivalent funds could purchase in more powerful, specialized server hardware.

Efforts to use commodity hardware for large computation tasks are orthogonal to the goal of treating computation power itself as a commodity. While clusters of commodity machines may be part of the underlying implementation of a commodity computation service, using specialized server hardware is also a viable option. When considering a service as a commodity, the methods used to offer the service are left to the provider, and innovation through proprietary techniques may prove a competitive advantage. From the customer's perspective, it is only the quality of the service rendered and the cost that matter.

\emph{Utility computing}, also called \emph{on-demand computing}, describes the business model of providing computing services and charging based on use of resources. This may apply to specialized services such as databases or web hosting platforms, or to any service where usage is metered and clients pay based on what they use rather than the capabilities that are available. By itself, utility computing answers only part of the problem. Customers are isolated from the fixed costs, risk of component failure, and administrative expenses associated with hardware ownership, but they may be restricted in the range of services offered or applications accomodated. Just as early mainframes required customized software development and incurred porting costs with each new iteration or change of vendor, utility computing systems subject clients to the tools and infrastructure requirements of their hosts.

\emph{Flexible commodity computation} is to a form of utility computing that allows standard commodity operating systems and tools as well as customized services to be deployed quickly and cheaply, where the basic environment provides commodity operating systems and tools rather than a specialized platform. Deployment costs are proportional to how far a client deviates from a standard, commodity environment, e.g., a standard Linux distribution, rather than the total amount of software their application requires to operate. By providing standard tools and standard environments, one flexible commodity computation service can be swapped for another without significant barriers such as porting costs and deployment costs, making computation itself a fungible commodity that can be used for a wide range of tasks, large and small.

\subsection{Commodity clusters}

Clusters have mainly been confined to solving ``embarrassingly parallel''\footnote{See \url{http://en.wikipedia.org/wiki/Embarrassingly_parallel}} problems such as search engines, scientific computing, data mining, and other parallel computing.

* single owners
* big tasks
* have to manage the hardware and the software

\subsubsection{Google}
\subsubsection{Beowulf clusters}
\subsubsection{Supercomputer 500 list}

\subsection{The GRID}

\subsubsection{Volunteer projects}
\subsubsection{Commercial and research projects}
\subsubsection{Planetlab}

\subsection{Web services}
\subsubsection{Thin clients}
\subsubsection{Mobile clients}

Mobile users \cite{demers}

\subsubsection{Hosted services}

email, payroll, tax prep

Storage outsourcing makes sense \cite{ng}.

\subsection{XenoServers}

\subsection{Replacing the machine room}

Hosted services are usually about the vendor: what can they do to drive business their way and keep it. Instead of picking the most lucrative services and offering them in full, how can we get rid of the machine room altogether and outsource the entire computational environment. This isn't about getting rid of desktops, it's more about database-driven apps, network services, computationally intensive tasks, and sporadic/intermittent demands.

\section{Service containers}

\subsection{Decoupling software and hardware management}

The provider of services and the trader of physical commodities resememble each other the most when the services of multiple providers are essentially interchangible, which requires agreement about not only what is to be done, but what is being acted upon. Cars of the same make and model can be repaired by a wide range of mechanics. Shipping firms can offer to transport a container of a specific size and weight between two points for a specific cost. Before the container was standardized, loading and unloading procedures would vary based on what was being shipped and how well it packed next to the goods of other customers. This would in turn affect the cost structure and tie together two services that are more efficient when seperated and optimized individually: loading and shipping.

Likewise, a web services platform may offer compelling services for its specific domains, allowing clients to host their web applications easier and cheaper than they could with their own hardware and software stack, but doing so would conflate two issues that could be better optimized individually: providing and managing the hardware resources, and managing the software infrastructure for web applications. The expertise required for these two parts of the problem may be quite different, and combining them forces clients to choose a package deal when they may be better served by independent choices. The skills of hardware managers may also be put to better use serving the needs of multiple software platforms at a larger scale, not just accomodating clients of a particular class of web services.

Any domain-specific middleware will necessarily be limited, and coupling the efficiency of a shared hardware infrastructure to a specific application domain will limit the economies of scale that could otherwise be achieved by more specialized providers. True commodity computation will divorce the application domain from the provision of a hosting platform, allowing specialists to excel in serving their respective functions. A platform that supports only a restricted domain of applications is offering a computation service, but it is not offering computation itself as its product. Attempting to port a service from a client's own machines to a hosted service provider to a competitor's platform may reveal how far each provider is from offering generic computation as a product.

The most flexible platform available to clients is wholly-owned and managed hardware. The PC has proved to be extremely adaptable and capable of hosting an enourmous range of applications. Giving clients a commodity hosting platform that approximates the flexibility of a standard machine gives them access to familiar tools and maximum latitude in designing their applications, without requiring them to conform to a specific middleware structure or use custom programming interfaces. It also protects the client from being locked-in to a single vendor through dependence on proprietary software.

Partitioning physical machines using a virtual machine manager and giving clients access to entire virtual machines pushes the dividing line between vendor- and client-management as close as possible to the hardware itself. Constraints still remain to retain control of the hardware and the ability to manage security concerns, but machine virtualization currently represents the state of the art in minimally decoupling control of the hardware from the concerns of the software stack.

\subsection{Decoupling unrelated services}

As hardware gets more capable, individual hosts can accomodate multiple applications. Organizations that wish to make efficient use of hardware investments must measure or estimate the requirements of each application and map them to machines in a way that maximizes the use of resources without overtaxing individual nodes. Managers are left with the choice to underutilize hardware resources, explicitly address load balancing in the applications, or manually allocation resources and rebalance as necessary. Each has its costs and its advantages.

Modern virtualization managers like Xen have low enough overhead to justify partitioning a machine even absent concerns about security and control of the machine \cite{barham}. By putting each application in its own virtual machine, the issue of hardware allocation can be seperated from the design and administration of the software itself. Instead of combining applications in an attempt to maximize hardware usage, a minimalist approach of assigning one application to one virtual machine becomes viable. VMs and the services they contain can be migrated as individual management units in response to runtime demand, without the explicit cooperation of the application.

Decoupling unrelated services seperates the problems of load balancing and maximizing resource utilization from the problems of software installation and deployment. Services contained in virtual machines become generic units that can be managed with generic tools, ignoring many of the intricacies of the actual software package.

Isolating services their own virtual machines also has the potential to increase security. While the same operating system and tool chain may be used, it can be stripped to include only those services and drivers necessary to support a single task. By being deployed with a minimal set of supporting software, a service can reduce its risk of being compromised by the flaws of unrelated services.

The disadvantage of this deployment model is that extra resources are consumed. In addition to the application software, operating systems and supporting libraries must be part of each service container. Overhead that is shared in a traditional environment is duplicated in each VM when services are partitioned in this way. This is a cost, but not one without reward; it buys flexibility and the potential for automation and simplification of management. Tayloring the runtime environment to the specific task can reduce the memory and CPU overhead without significant reengineering, and suitable storage strategies can reduce the storage redundancy that otherwise results from increasing the number of virtual machines complete with operating systems.

\subsection{Supporting commodity tools}

Any suitably designed framework can seperate the management and control of hardware from that of software; this is one of the basic functions of operating systems. Similarly, balancing the demands of applications against the capabilities of hardware is a recurring theme in system design. Achieving both of these aims while permitting the use of a wide range of standard, commodity tools and operating systems precludes custom frameworks, however. Virtualization and a discipline of packaging applications into minimal service containers brings these capabilities to existing applications without the expense of porting to a new software platform.

Using commodity software when appropriate brings many of the same advantages as commodity hardware. Commodity operating systems and tools are cheap, powerful, and under constant development, so features and improvements accrue over time. Just as using commodity hardware allows users to benefit from the scale and competition of a thriving market, relying on commodity software gives access to the benefits of industry-wide testing and development efforts driven by competition and a large, demanding user base.

An important characteristic of commodity software tools is that they are widely used, and the most popular can be easily identified. Even without a formal process, standards emerge over time and change slowly, both in proprietary and open source software communities. Computation providers can streamline deployment and reduce costs by offering a small set of file system images based around \emph{de facto} standards, complete with an operating system and standard tools. Clients can then customize an image to support a specific service, and deploy it with little additional effort. The setup time needed, bandwidth consumed, and storage required to customize the image are related to the degree of customization required, not to the overall complexity of the service. As supporting tools become more sophisticated and capable, and as the base of standard software evolves, more intricate services can be deployed without increasing the deployment costs.

The users best served by these base images are those whose needs are met entirely by commodity software. Deploying a DNS server or a web server requires little more than configuration and some content, all of which can be transferred using standard tools on a private virtual machine. Offering standard base images as an option does not restrict clients to what is provided, however. The architecture favors the use of standard tools, but it does not prevent users from starting from scratch. As is true in many product domains, departing from the standard is discouraged only by the higher cost. As is also true in many kinds of product fabrication, making a custom image incurs some one-time costs; using the image as the base for many service instances makes it possible to amortize that cost.

\section{Service clusters}

\subsection{Definition of a service cluster}

One of the enduring goals of systems research is to provide a good platform for running applications. Even early systems such as Multics were explicitly intended as infrastructure for higher-level computing services, seeking ``continuous operation in a utility-like manner, with flexible remote access,'' with requirements such as ``convenience of manipulating data and program files, ease of controlling processes during execution and above all, protection of private files and isolation of independent processes'' \cite{corbato}. Four decades has seen much progress, but similar goals are still applicable.

The first part of the problem of commoditizing computation is packaging tasks into managable units. As argued above, service containers are a flexible way to isolate applications from the machines that host them and from unrelated tasks with which they may share hardware. While service containers can be deployed on individual machines, networks of workstations, or other groups of hardware, it is in cluster environments that they find their natural home. \emph{Service clusters} are clusters of commodity machines managed centrally to support the deployment of arbitrary service containers, either as isolated instances or as groups of interacting services.

\subsection{Economies of scale}

The most obvious benefits of hardware clusters are related to scale. Quantity purchases generally lead to better prices, and dedicated facilities can be streamlined for a single purpose to eliminate waste, e.g., temperature control, lighting, and physical layout can be optimized entirely for the hosting machines, rather than for a mixed environment of machines and people. Fixed costs can be amortized over large numbers of nodes, and running costs can be negotiated for bulk quantities.

Scale also makes it possible to devote resources to system design that would be impractical for smaller deployments. Staff can devote all their time to managing the lower levels of the computation stack---from hardware up to the virtual machine---and to optimizing the platform without specific applications in mind. Facilities can also be located away from client buildings to take advantage of favorable business environments, high-speed internet connections, and available staff. Scale and access to a wide range of clients also increases the potential rewards for improving efficiency and service.

Hardware can be added to clusters incrementally, which eliminates the need for an accurate forecast of the lifetime demands of the system. In addition, clusters of machines can achieve much greater overall scale than even the largest single machines. Scalability over time and absolute scalability are also features of non-clustered distributed systems, but those lack the high-speed interconnects and coordinated administration possible in managed environments.

The use of commodity hardware also allows rapid machine acquisition and cheap prices. Incremental scaling means that newly added hardware can always take advantage of the best price/performance ratios and benefit from the constant downward pressure on prices in a competitive market \cite{fox}. Commodity disks are relatively unreliable, but they are also large and cheap and offer a good source for storage capacity \cite{patterson,warfield} that comes standard with most machines.

The independence of nodes in a cluster offers redundancy that can be exploited for both reliability and availability. This is necessary not only to exceed the expectations of a single system, but to match it as well. Having many parts that can fail independently offers a much higher probability that at least one will fail than that any single component will fail, and a cluster that does not tolerate some component failures will quickly become unusable \cite{birrell93}.

\subsubsection{Heterogeneous workloads}

Oversubscribing capacity is a common practice for businesses that offer a fixed level of service, but expect some users to use only part of what is offered. Airlines overbook flights with the expectation that some passengers will forfeit their places, allowing the airline to capture revenue based on the promised service, not the service delivered. Some web hosting services oversubscribe their hardware capacity, a practice that allows them to provision based on expected average use rather than maximum potential use.

Since clusters can be expanded incrementally, observed average use can become the metric for established providers, especially at large scale. It may be prudent to hold some reserve capacity to handle bursts in usage, but even burstiness becomes more predictible at large scales. The more diverse the services using a cluster, the less likely an external event will trigger a burst in usage that overwhelms the capacity of the entire cluster. Heterogeneity and varied workloads may be difficult to manage at a small scale because they are difficult to anticipate and plan for, but at larger scales their uncorrelated fluctuations become a benefit.

Random variations in activity must be accomodated, but periodic fluctuactions in demand can sometimes be planned for and exploited to make better use of resources. Predictable events like business hours, holidays, academic calendars, and other stable cycles can significantly influence some service workloads. Planning for these by pairing complementary services in a dynamically-configured cluster can help avoid idle resources and reduce expenses. Systems used heavily during business hours could share with systems used by game servers, assuming that the latter are used more during leisure hours than on company time.

\subsection{Central management}

Clustering groups of machines together enhances scalability and resilence to failed components compared to a single system, but it does not simplify application design. New failure modes, networked interconnects, the lack of shared busses, and the lack of shared memory and processor control all change the way systems must be designed. The simplicity of a single system is lost in a cluster, but some of its features can be retained or at least emulated. Clusters, unlike wide-area distributed systems, are generally physically close to each other and managed under a single administrative domain. Physical security and the level of trust placed in each node is increased as a result.

Centralized administration and high-speed communication (via shared memory and IPC) are two advantages of traditional servers. Clusters typically put components in a single location with high-speed local networking and a secure physical location. The machines are all owned and administered by a single organization and can be built with appropriate cooling systems and redundant power supplies. Systems designed to prevent or avoid any centralized control usually do so for privacy or legal reasons, neither of which is compelling in the service cluster environment. On the contrary, some degree of central administration is an essential feature of service clusters. The owner of the cluster needs to control access and admission to the service pool as well as monitor the services that are running to ensure that they do not exceed their alloted resources. Given physical proximity and central control, central administration adds convenience without introducing unnecessary penalties in flexibility.

To further emulate the desirable characteristics of centralized servers, clusters must ensure global access to data in the storage system from any constituent node. This is best achieved through a single, global name space with a completely coherent view of all files in the cluster. Trading coherence for performance represents a failure of global access, as concurrent services effectively create private views of the data, requiring a seperate mechanism for restoring the consistency that the file system violated \cite{birrell93}. While partial coherence is sufficient for some applications, it exposes differences between local and distributed systems and weakens the guarantees that the file system offers. This requires planning for all system designers, even those that ultimately determine that the weaker guarantee can be safely ignored \cite{waldo}.

Services running on a single server all fail together if the server fails. Correlated failure is generally considered a problem in distributed systems, but it can also simplify the requirements of failure recovery at least in the common case. When a single server fails, the file system is generally restored to a consistent state before its services are restarted and allowed to recover. If the file system fails while the service continues, this is often considered a similarly catastrophic failure of the system, no better than the two parts failing together. In a virtual machine environmenet, the distributed file system manager can be located on the same physical machine as the service that depends on it, restoring this hardware correlation between the two components. Note that the virtual machine monitor already represents a single point of failure for an entire physical machine; adding the file system manager as a second critical software component that cannot be recovered without a restart of the node only enlarges the set of critical tools, it does not create it. The failure of a client service does not enjoy the same priviledged status---the rest of the node must be able to continue functioning without it.

Service clusters bring together a wide range of clients under central management, making it possible to monitor and model the behavior of unrelated services and plan resource allocation with more information than a single client could provide. Unrelated clients may have complementary resource requirements, e.g., one demands many CPU cycles and another much memory, which a cluster manager can detect and exploit in mapping services to physical nodes. This applies to characteristics observed over time as well, such as cycles of demand driven by external factors such as time of day or day of the week. Putting a wide range of services from a wide range of clients together in a single cluster allows administrators to make decisions informed by pertinent, observed data, coaxing out optimizations that would not be available to clients acting on their own.

\subsection{Supporting a software ecosystem}

Service clusters have the ambitious goal of replacing the machine room for many organizations with hired resources on a commodity service. To achieve this they must support a similar range of activities and offer tangible benefits compared to privately owned and managed hardware. For some, commodity computation functions as a direct replacement for owned and managed hardware resources. Maximum flexibility and access to a full set of standard and customized tools with minimum overhead serves these needs.

Packaging computation tasks as service containers yields advantages for administration that could be equally useful in private machine rooms. Isolating applications in fine-grained protection domains and maximizing resource utilitization through allocation and reallocation at the service level benefits hardware owners and application writers alike. A platform with that level of flexibility and control as a resource for hire offers new possibilities for service providers that narrower service offerings do not.

With low-level services available, third parties can offer intermediate levels of service more appropriate for specific uses. Instead of offering packaged applications, sending consultants to assist with deployment, or hosting software themselves on their own hardware, software vendors can contract their services on a computation platform. Web hosting, groupware, email hosting, payroll services, etc., all exist currently as managed services, coupling the software services with the hardware services. With a low-level computation platform available, clients can seperately negotiate each aspect of a hosted service---the hardware hosting and the software management---and retain greater control over their own data.

In addition to end-user services, a service cluster economy could support an entire ecosystem of intermediate services. While standard software installations as base images form an important part of flexible commodity computation, they will not be appropriate for everyone. Some clients may wish to purchase database services from a vendor, running on the same service cluster as the client but managed by a third party. Scalable and widely-distributed web hosting could be offered by a vendor that buys hardware resources as needed from a range of service clusters, then sells simple packages to individual clients. Expertise in building distributed services and managing complex software has value that need not be coupled with hardware management.

In an ecosystem of hosted software, service clusters are the base environment upon which other services build. Clients may require only a single service container, or they may hire a large amount of capacity, either to fill their own needs or to export their own higher-level services to other users. To support these different usage patterns, service clusters must address the needs of shared groups of services as well as services in isolation.

\section{Storage for service clusters}

The design and intended applications of service clusters put specific demands on the storage system. While many of the storage requirements have been explored individually in other settings, the combination is unique, and existing systems only partially address the needs of the environment.

The major relevant characterstics of a service cluster are the cluster architecture, the reliance on commodity hardware, the support for commodity software, the split in control between the operators of the cluster and the owners of individual services, the expectation of singleton services and cooperating groups of services, and the use of virtual machines as units of management.

Note: What is this section about? There's a design requirements section in the next chapter; should they be merged? If so, where should it live?

\section{Summary}
