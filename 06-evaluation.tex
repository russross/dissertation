\chapter{Evaluation}\label{cha:evaluation}

synthetic workloads are a hard problem \cite{ganger95}

\section{Experimental setup}

\subsection{Test machines}

\subsection{Benchmarks}

\subsubsection{Linux source tree}
untar -- write test
tar > /dev/null -- read test
rsync -- 2nd read test

\subsubsection{Bonnie}

\section{Performance}

If the goal is to get stuff as local as possible, quantify the benefits of achieving that.

\subsection{Architectural costs}

userspace nfs
userspace 9p
envoy local dom0
envoy local domU
envoy remote dom0
envoy remote domU

cache: cold vs warm vs hot

\begin{itemize}
\item standard benchmarks verses userspace 9p. Postmark \cite{katcher}.
\item same domain verses different domain verses different machine
\item read/write verses metadata heavy
\item persistent cache: cold, warm, hot, disabled
\end{itemize}

\subsection{Fancy features}
\subsubsection{Forks}
\subsubsection{Snapshots}
\subsubsection{Lease migration}
\subsubsection{2-way contention: owner verses remote host performance}
i.e., what's to be gained by rebalancing (besides load balancing). isn't this covered by the baseline stuff?

\section{Scaling}
\subsection{Independence of private images}
\subsection{Degredation of a single host with many clients}
Shared image, and also private images on one machine

\section{Dynamic behavior}

Test dynamic territory management

\subsection{Example scenarios}

shared image with (independent) home directories

2 log files in one directory

producer-consumer

\subsection{Sharing application}
distcc or something else that shares in a complex but predictable fashion

\subsection{Synthetic workloads}

probabalistic traffic driven to overlapping areas

\section{Case studies}

\subsection{Service deployment}
show sharing: boot one VM from cold cache, boot another based on same template

\subsection{Quantifying sharing}

SUSE10 upgrade, install services, etc.

\section{Summary}