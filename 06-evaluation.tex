\chapter{Evaluation}\label{cha:evaluation}
synthetic workloads are a hard problem \cite{ganger95}
\section{Experimental setup}

\subsection{Test machines}

\subsection{Benchmarks}

\subsubsection{Bonnie}
\subsubsection{Postmark}

\subsubsection{Linux source tree}
untar -- write test
tar > /dev/null -- read test
rsync -- 2nd read test
find

On druid-1 (local envoy control, local storage is master, druid \& skiing storage)
results-untar-same-machine.txt



\section{Baseline performance}

If the goal is to get stuff as local as possible, quantify the benefits of achieving that.

\subsection{Comparisons}

ext3
userspace nfs
kernel nfs
userspace 9p
envoy same VM
envoy same machine
envoy remote machine

cache: cold vs warm vs hot

\begin{itemize}
\item standard benchmarks verses userspace 9p. Postmark \cite{katcher}.
\item same domain verses different domain verses different machine
\item read/write verses metadata heavy
\item persistent cache: cold, warm, hot, disabled
\end{itemize}

\section{Fancy features}
\subsection{Snapshots}
\subsection{Lease migration}
\subsection{2-way contention: owner verses remote host performance}
Performance gains from rebalancing (besides load balancing).

\section{Scaling}
\subsection{Degredation of a single host with many clients}
Shared image, and also private images on one machine
\subsection{Independence of private images}
\subsection{Loading the storage layer}

\section{Case studies}
\subsection{Service deployment}
\subsection{Quantifying sharing (FC3 upgrade, install services, etc.)}

\section{Summary}