% Required summary page
% Must include name, college, title, date, and the abstract
% approximately 300 words

% for the separate summary:

% \clearpage
% {
% \thispagestyle{empty}
% \begin{center}
% {
%     \Large \thesistitlebig
% 
%     \bigskip
% 
%     \textit{\large Summary}
% }
% \end{center}
% 
% \hfill Russell Glen Ross\\
% December 2006 \hfill Wolfson College\\
% }

% for the dissertation:

\chapter*{Summary}

% the summary

\noindent Standards in the computer industry have made basic components and entire architectures into commodities, and commodity hardware is increasingly being used for the heavy lifting formerly reserved for specialised platforms. Now software and services are following. Modern updates to virtualization technology make it practical to subdivide commodity servers and manage groups of heterogeneous services using commodity operating systems and tools, so services can be packaged and managed independent of the hardware on which they run. Computation as a commodity is soon to follow, moving beyond the specialised applications typical of today's utility computing.

In this dissertation, I argue for the adoption of \emph{service clusters}---clusters of commodity machines under central control, but running services in virtual machines for arbitrary, untrusted clients---as the basic building block for an economy of \emph{flexible commodity computation}. I outline the requirements this platform imposes on its storage system and argue that they are necessary for service clusters to be practical, but are not found in existing systems.

Next I introduce Envoy, a distributed file system for service clusters. In addition to meeting the needs of a new environment, Envoy introduces a novel file distribution scheme that organises metadata and cache management according to runtime demand. In effect, the file system is partitioned and control of each part given to the client that uses it the most; that client in turn acts as a server with caching for other clients that require concurrent access. Scalability is limited only by runtime contention, and clients share a perfectly consistent cache distributed across the cluster. As usage patterns change, the partition boundaries are updated dynamically, with urgent changes made quickly and more minor optimisations made over a longer period of time.

Experiments with the Envoy prototype demonstrate that service clusters can support cheap and rapid deployment of services, from isolated instances to groups of cooperating components with shared storage demands.
