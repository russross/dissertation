\chapter{Conclusion}\label{cha:conclusion}

This dissertation concludes with a discussion of opportunities for future research and a summary of the work presented here.

\section{Future Work}

Storage clusters are only partially specified in this work, and additional research is necessary to make them a reality. Even ignoring the storage layer, virtual machine tools, cluster management tools, and economic modelling that are outside the scope of this dissertation, several aspects of Envoy's design suggest avenues for future research.

\subsection{Caching}

Envoy eschews client-side caching in favour of a unified node-level cache. This works well for some workloads, but poorly for others, particularly for memory-mapped files. Besides being part of Envoy's cache consistency mechanism, this design allows consolidation of redundant cache entries that would otherwise exist on multiple clients. While template images encourage object sharing, many objects will still be used privately by only one client at a time, where issues of consistency and consolidation do not apply. Two questions remain unanswered: how much of the performance penalty from moving the cache outside the client's VM can be optimised away, and how could a hybrid cache that uses client-side caching with token-passing for some data and a consolidated cache for the rest be constructed to balance competing concerns of individual client performance and aggregate cache capacity and performance.

\subsection{Territory partitioning}

Envoy uses a simple greedy algorithm to decide on territory boundary changes, with a unique time-based mechanism to promote stability while still permitting minor optimisations over time. Further testing with realistic workloads would improve understanding of the algorithm's behaviour, and may suggest improvements. The existing implementation could be enhanced to make more sophisticated recommendations, for example keeping a subtree and migrating the rest of the territory whereas now it can only recommend the converse. Also, finding optimal configuration parameters that balance the benefits of locality with the costs of moving to an unprimed cache in realistic conditions remains future work.

\subsection{Read-only territory grants}

Territories in Envoy divide and subdivide file system images to align ownership with access. Currently, a single owner monopolises each territory and maintains its cache, resulting in performance penalties for remote clients. Implicit sharing is available through image forking, but since most file data is never modified, additional sharing of territories could be enabled. Territories could be extended to permit read-only grants to multiple nodes, which are revoked and consolidated when write operations are requested, effectively extending write-exclusive/read-shared locks to the territory level. It remains to be seen whether or not this would provide significant benefits for realistic workloads.

\subsection{Storage-informed load balancing}

Envoy entrusts territory management decisions to the current territory owners because they have the most information to make good allocation decisions. Similarly, the file system in a service cluster would be a good place to collect information about client behaviour to inform load balancing and service placement decisions. Envoy's structure already detects clients with overlapping interests in active images, and could also infer common interest in objects shared through image forking. Gathering and using this information could make it possible to optimise system-wide service layout to minimise storage-layer demand and enhance the value of shared caching, as well as increasing the aggregate performance and capacity of the service cluster.

\section{Summary}

This dissertation argues for the commoditisation of computation services using virtual machines as containers hosted on clusters built from commodity hardware. In particular, it addresses the storage demands of this environment, a need not adequately filled by existing systems. After an introduction to the problem in \charef{cha:introduction}, \charef{cha:background} discusses relevant background work in commodity computing, machine virtualization, and storage systems.

\charef{cha:motivation} describes the proposed environment in more detail, and argues its merits as well as describing its requirements. \emph{Service containers} are defined as virtual machines customised from template images of commodity software, and \emph{service clusters} are cluster environments optimised for deploying client services in a flexible way. This combination permits management of untrusted client software in ways that maximise the use of hardware resources and separate hardware management from software deployment.

The Envoy file system is introduced in \charef{cha:design} and its implementation described in \charef{cha:implementation}. Envoy runs in a trusted virtual machine on each node, which exports a standard client-server file system interface to locally-hosted clients. A global namespace tree includes file system images that can be managed and shared under client control. Control of images is given to the envoy node on the same host as the client, and control of shared images is partitioned into \emph{territories} by a dynamic process that responds to runtime demand. In a stable state, nodes need to communicate with each other only when the interests of their respective clients overlap. Lightweight snapshot and fork operations allow simple image management, and encourage implicit sharing of underlying objects even when no relationship exists between clients.

Evaluation of the Envoy prototype in \charef{cha:evaluation} demonstrates that baseline performance in the most common configuration is comparable to simple client-server file systems. The forwarding architecture imposes a 33\% overhead for remote requests in addition to added network latencies, making it beneficial to optimise territory boundaries but not overly expensive to tolerate sub-optimal layouts with light traffic. Caching, both in-memory and on-disk, improves read performance, though on-disk caching comes at a minor cost for write operations. The dynamic territory algorithm successfully optimises common sharing scenarios. Booting multiple services from forks of a common template image yields implicit object sharing that results in a significant reduction of traffic to the storage layer.

Envoy supports the storage needs of service clusters, allowing flexible and efficient management of storage and optimising for expected usage patterns, with an architecture that promotes scalability by localising storage management and encouraging cache sharing.
