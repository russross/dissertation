\section{NYSE and time-based flexibility}

At 9:30~a.m.\ the New York Stock Exchange opens for trading and the computing demands of many financial services companies are focused on the fluctuations of the markets until the closing bell rings at 4:00~p.m. In the hours until the markets open again, the resources of these companies may turn to intensive jobs---such as simulations---for the company itself, or to handling the processing needs of clients. The applications and data sets involved may be completely unrelated to those used during the daytime hours.

Standard business hours drive many other computing patterns as well. A web site will have the highest traffic load when its target audience is awake and in front of a computer. For workers using personal time, this may mean a traffic surge in the morning, another around lunchtime, and a third during the early evening. For email and other services used by businesses, the traffic curve follows business hours, possibly in many time zones. In contrast, game hosts are busier during leisure hours and see many clients otherwise engaged when corporate offices open their doors.

As internet services become more sophisticated, human clients are not the only ones that drive resource usage. Mobile agents and other automated services are increasingly becoming clients as well as servers. Amazon recently unveiled a service to open their storage platform directly for use by others for web services. Major internet services like eBay and Google offer APIs for third parties to interact with their services using custom software, rather than just their usual web interfaces for human clients.

All of these examples benefit from the simple and rapid deployment and management of software services in a range of environments. Recent years have seen a renaissance of machine virtualization technology, this time in cheap, commodity hardware.

In this chapter I start by defining \emph{flexible commodity computation} and discussing some examples that motivate its adoption. I then discuss existing commodity computing systems and how they succeed at exploiting commodity systems but fail at making general-purpose computation a true commodity. I argue that the combination of virtual machine managers and clusters of commodity hardware offers a potent platform for service deployment, which I call \emph{service clusters}. I further argue that existing storage systems are not well suited for this environment when used as a platform for flexible commodity computation, so a new set of requirements and a storage system to match them are called for.

\section{Service clusters as a platform using VMs}

Hardware virtualization is one of the more potent tools available for achieving these ends. Virtual memory systems are a standard tool for isolating processes managed by operating systems, and virtualizing entire hardware platforms extends the ability to partition resources to the operating system level, which is a more flexible unit of management \cite{hand}. Once confined mainly to specialized server hardware, machine virtualization is emerging as an important general purpose management tool. The Xen hypervisor \cite{barham} was the first to make virtualization efficient for commodity hardware, especially for I/O intensive applications.

Using virtual machines makes it easier to manage OSes \cite{chen}. Gartner thinks virtualization is the way to go \cite{bittman}.

\section{Distribution}

Three basic architectures are available. A centralized server model is the simplest and satisfies the mobility requirement, but it doesn't scale well and suffers from a single point of failure. This model has been extended quite successfully using RAIDs and SANs on the back end to increase performance and robustness, and by caching on the client end to reduce load, but it is still limited. Ultimately, all decisions are moderated by a single host which must communicate directly with all clients, so at larger scale the network bandwidth becomes a problem as well.

At the other extreme, coordinations is pushed completely to the clients. This is essentially the centralized model turned upside down. Instead of many clients talking to a single server, a single client must communicate with many servers. Instead of one congestion point, every client becomes overwhelmed with bookkeeping. In practice, some structure can be introduced so that the system isn't a full $n$-way clique---the host for each object of storage may become the mediator for access to that object, for example---but making every participant a symmetric peer is mostly useful for privacy or other non-technical concerns.

A more practical approach is to emulate a centralized architecture as much as possible, since this provides the simplest model with the least administrative overhead, and carve up the problem enough to achieve the desired scaling properties. The two extremes offer the greatest conceptual purity, but the practical benefits are to be had somewhere between them. Most P2P services now introduce some kind of a hierarchy that designates some hosts as ``superpeers'' or ``supernodes'' that act as aggregation points for a manageable set of participants. These distinguished hosts only need to communicate directly with other similarly endowed hosts, potentially reducing the visible size of the network by orders of magnitude. The world is small enough and computers big enough that a few orders of magnitude is enough reduction to make many distributed problems tractable.

Another way of viewing this problem is to consider how many manually administered systems are often implemented today. NFS \cite{callaghan} is still one of the most widely-used systems, and a workstation environment may include a series of NFS servers to provide storage for a larger number of client workstations. Each server will be backed by a RAID \cite{patterson} system for improved performance and reliability, and the administrators may divide the file system namespace in order to balance average demand over the different servers. While most file access in uncontended, having a single server that serves all interested clients makes coherence easy (ignoring the effects of caching).

An interesting thought exercise is to consider the same set of workstations locking into a long-term steady-state behavior, and then asking the systems administrator to hand configure the system based on those usage patterns. One reasonable possibility would be to locate files or branches of the namespace tree used exclusively by a single workstation on that workstation. When two or more users require the same file or set of files, it could be managed by the workstation that uses it the most and exported via NFS (or a similar client-server file system protocol) to the other participants. Since contention is relatively rare, this would reduce network traffic for the storage system to a low level for as long as the hypothetical steady-state continues. One of the goals of the Envoy design is to capture this approximate topology.

In service clusters, physical hosts are divided into roughly 10s of services, which may be transient, untrusted, and unreliable. This increases the number of participants requiring access to the storage system by the same factor and exposes difficult security problems. One of the principle benefits of isolating services in individual virtual machines is in reducing the impact of services that fail, either benignly or maliciously, and trusting the administration of the file system to the services negates this benefit.

The environment includes a trusted management domain on each server, however, which introduces a natural aggregation point and a way to sidestep many of those problems.

\section{Supporting services}

Live migration of virtual machines \cite{clark} is essential for load balancing and uninterrupted service in the presence of hardware servicing. To support migration, a file system cannot be tied to a single physical machine; any given file system image must be as mobile as the service that relies on it.

One of the attractive features of running individual services in virtual machines is the flexibility of management this model offers. Services can be decoupled, instances can be created and destroyed easily in response to need, and a new service can be deployed without dedicating a new machine to it, removing an old service to make room, or studying the potential interactions that would occur if a shared server was the only option. To this end, lightweight operations are required to instantiate and clone services.

The unit of management is an entire operating system on a virtual machine. Installing an operating system for each service instance would be prohibitively slow, labour intensive, and wasteful of space. A typical installation of Linux uses hundreds of megabytes or gigabytes of space, and an automated installation is unlikely to give the right set of installed packages and running daemons, so some customization will be necessary.

Instead of trying to automate installations, Envoy offers lightweight operations to fork a file system.

\section{Overstudied storage systems}

Storage systems have been studied and documented extensively in the systems literature. One could argue that everything worth doing has been done, or alternatively one could assert that the continuing research is evidence that everyone has failed so far, with the truth probably residing somewhere between these two extremes. Storage is an essential component of most systems, though, and the requirements placed on them are almost as various as the solutions that have been proposed.

In this chapter I survey the most important and widely known storage systems, with an emphasis on those that are relevent to service clusters and similar in design to Envoy. I describe some of the major design tradeoffs that emerge and identify the characteristics of service clusters that make existing systems unsuitable or less than ideal for this emerging environment as I justify the need for another storage system.

\section{Denali and Terra}

Denali \cite{whitaker02} is also designed to support deployment of untrusted services on shared hardware, but it breaks binary compatibility to support a narrow range of services in large numbers. While suitable for some types of utility computing, it is not flexible enough for general-purpose commodity computation. Terra \cite{garfinkel} also addresses a different problem by securing guest services using hardware extensions to provide a trusted environment.