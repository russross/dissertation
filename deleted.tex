\section{NYSE and time-based flexibility}

At 9:30~a.m.\ the New York Stock Exchange opens for trading and the computing demands of many financial services companies are focused on the fluctuations of the markets until the closing bell rings at 4:00~p.m. In the hours until the markets open again, the resources of these companies may turn to intensive jobs---such as simulations---for the company itself, or to handling the processing needs of clients. The applications and data sets involved may be completely unrelated to those used during the daytime hours.

Standard business hours drive many other computing patterns as well. A web site will have the highest traffic load when its target audience is awake and in front of a computer. For workers using personal time, this may mean a traffic surge in the morning, another around lunchtime, and a third during the early evening. For email and other services used by businesses, the traffic curve follows business hours, possibly in many time zones. In contrast, game hosts are busier during leisure hours and see many clients otherwise engaged when corporate offices open their doors.

As internet services become more sophisticated, human clients are not the only ones that drive resource usage. Mobile agents and other automated services are increasingly becoming clients as well as servers. Amazon recently unveiled a service to open their storage platform directly for use by others for web services. Major internet services like eBay and Google offer APIs for third parties to interact with their services using custom software, rather than just their usual web interfaces for human clients.

All of these examples benefit from the simple and rapid deployment and management of software services in a range of environments. Recent years have seen a renaissance of machine virtualization technology, this time in cheap, commodity hardware.

In this chapter I start by defining \emph{flexible commodity computation} and discussing some examples that motivate its adoption. I then discuss existing commodity computing systems and how they succeed at exploiting commodity systems but fail at making general-purpose computation a true commodity. I argue that the combination of virtual machine managers and clusters of commodity hardware offers a potent platform for service deployment, which I call \emph{service clusters}. I further argue that existing storage systems are not well suited for this environment when used as a platform for flexible commodity computation, so a new set of requirements and a storage system to match them are called for.

\section{Service clusters as a platform using VMs}

Hardware virtualization is one of the more potent tools available for achieving these ends. Virtual memory systems are a standard tool for isolating processes managed by operating systems, and virtualizing entire hardware platforms extends the ability to partition resources to the operating system level, which is a more flexible unit of management \cite{hand}. Once confined mainly to specialized server hardware, machine virtualization is emerging as an important general purpose management tool. The Xen hypervisor \cite{barham} was the first to make virtualization efficient for commodity hardware, especially for I/O intensive applications.

Using virtual machines makes it easier to manage OSes \cite{chen}. Gartner thinks virtualization is the way to go \cite{bittman}.
