\documentclass[a4paper]{article}
\usepackage[T1]{fontenc}
\usepackage[utf8]{inputenc}
\usepackage[margin=1in]{geometry}
\usepackage{sabon}

\bibliographystyle{plain}

\title{Dissertation Structure}
\author{Russ Ross}
\date{August 2006}

\begin{document}

\maketitle

\section{Introduction}
Thesis: virtual machine clusters have new storage requirements
\begin{itemize}
\item lots of images: 1,000s of nodes and 10s of VMs per node
\item migration: no such thing as local storage
\item deployment: image forks are essential to make lightweight service deployment practical
\item commodity disks: VMs on commodity hardware are driving this area, using the big cheap (and unreliable) disks is a no brainer
\item topology: admin domain as aggregation point for persistent caching and buffering
\item service model: lightweight private images mixed with synchronous shared images
\item security: admin domain lets us trust servers, private images means we should give as much control to clients as possible
\item snapshots: rollback for debugging and forensics
\item scalability: burden for private images should fall completely on the local node, shared images should be limited only by (actual) contention
\end{itemize}

\subsection{Virtual machines}
\subsection{Motivation}
\subsubsection{Things that a VM cluster calls for}
\begin{itemize}
\item virtual machine clusters---admin domain as aggregator and secure manager
\item mobile VMs
\item stock base images with customization---rapid deployment of services
\item commodity disks
\item snapshots and forks
\end{itemize}

\subsubsection{Desirable properties that the envoy model achieves}
\begin{itemize}
\item distribute only when there's a reason; favor centralization when practical
\item perfectly consistent persistent caching
\item local impact---heavy users bear most of the load, non-users none of it
\item serve from local machine cache when uncontended, NFS-like when shared: requests never require topology changes
\item in steady state, coordination based on actual contention, not potential contention
\item simple security model that maps well to familiar Unix semantics
\item private images act like local images, shared images scale gracefully
\end{itemize}


\section{Background}

\subsection{Xen}
\subsection{XenoServers}
\subsection{CoWNFS}
\subsection{Parallax}
\subsection{9p}
\subsection{Related works}
\begin{itemize}
\item Network file systems (NFS, Sprite, AFS, Coda, 9p, Venti)
\item P2P file systems (content addressed, DHT)
\item Distributed (xFS)
\item Cluster (object layers, big shared volumes, GFS, Lustre)
\item Block device servers (SANs, Parallax, FAB)
\item File distribution systems (Bittorrent, Gnutella)
\item VM work (Ventana, CoWNFS, Parallax, VMWare?)
\end{itemize}


\section{Design}

\subsection{User interface tour}
\subsection{High-level architecture overview}
\subsection{Storage layer}
\subsection{Envoy layer}
\subsection{Lease migration}
\subsection{Forks and snapshots}
\subsection{Security}
\subsection{Recovery}
\subsection{Deleting snapshots}

\section{The Envoy prototype}

\subsection{}

\section{Evaluation}

\subsection{Baseline comparisons to userspace 9p server}
\subsection{}

\section{Conclusion and future work}
\begin{itemize}
\item Envoy model in other settings
\item Readonly leases
\item Other partitioning/lease migration strategies
\item Implementation in NFSv4?
\end{itemize}

\nocite{*}
\bibliography{fs}

\end{document}