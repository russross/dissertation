\chapter{Background}

\section{Commodity computing}


Similar idea as network appliances: \cite{sapuntzakis03}. SoftUDC: software-based data center \cite{kallahalla}. Utilification \cite{wilkes04}.

\subsection{Network of workstations}
Networks of workstations (NOW)\cite{anderson95a}

\subsection{Clusters}

Embarassingly parallel problems

CARD for cluster monitoring\cite{anderson97}

Fast crash recovery verses redundancy\cite{baker94}

Partition cloning \cite{rauch}

TranSend was written for and the Inktomi search engine adapted to a layered cluster platform for scalable, homogeneous network services. Services composed of a few distinct types of actors could be managed by a common platform that provided monitoring and allocation of the component parts over the cluster. This type of platform exemplifies many of the advantages of clusters for network services, including load balancing, incremental scaling, fault tolerance, and high availability, but the implementation was limited to ``embarassingly parallel'' problems with trusted components and a custom state management system backed by a commercial SAN \cite{fox}. The authors were able to demonstrate that scalability was limited more by bandwidth into the cluster than by communication within it or coordination overhead.

\subsection{GRID}
\cite{zhao04}

\subsection{Utility computing}
\subsection{Hosted services}

\section{Virtualization}

\subsection{Virtual machine monitors}
IBM System/370 \cite{gum}
Disco
Denali
Xen \cite{barham}
Live migration \cite{clark} \cite{sapuntzakis02}
Terra \cite{garfinkel}
VMWare
vMatrix \cite{awadallah}
IBM Managed Hosting

\subsection{XenoServers}
\cite{kotsovinos04a,kotsovinos04b}

Ventana \cite{pfaff}

\section{Storage systems}

Storage is a fundamental component of all general-purpose computer systems. A combination of high storage density, random access, and low cost have made magnetic storage on rotating platters the dominant medium for durable storage. The time to access a random byte of data from a disk is typically measured in milliseconds, while for DRAM this time drops roughly six orders of magnitude to a nanosecond scale, and that gap is continuing to grow.

Because disks are so slow, storage systems are designed around the goal of accessing the disk as little as possible, and favoring sequential access to avoid costly seek delays. Effective use of cache is vital to this goal, and specialized storage systems tuned to specific system architectures and expected workloads are worthwhile because of the potential speed gains over more general systems.

Storage systems have been studied extensively, and one purpose of this section is to survey related work that has influenced this dissertation. A second purpose relates to the thesis of this work, which argues that a new environment justifies a new storage architecture. The works discussed here are presented in the context of the environments that inspired them, in part to establish a pattern of symbiosis between storage architectures and computation environments, a pattern at the heart of the present work. Envoy, the file system introduced in this dissertation, is discussed briefly in this section to relate it to prior work, but details of its design are reserved for later chapters.

\subsection{Local file systems}

Disks are mechanical devices and their performance is limited primarily by the need to physically position the disk head over the correct part of the platter to access data. File systems designed for local disks achieve performance by minimizing the physical movement required. Disk access can be reduced through caching, and hardware latency can be minimized by arranging data on the disk to minimize movement. Correctness and reliability tend to trump performance as design considerations, however, and the wide variety of possible workloads makes it difficult to pick clear winners from competing designs.

The Berkeley Fast File System (FFS) was the first to optimize data layout for performance by clustering related information. Block metadata is distributed across the disk to be near the file data it describes, file metadata is grouped for files in the same directory, and blocks in the same file are grouped whenever possible \cite{mckusick}. \emph{Clustering}, as this design is called, exploits concepts that extend beyond local file systems: the same expectations of correlated access can be exploited to reduce latency when accessing data across the network.

File system tracing reveals how files are accessed in real systems \cite{ousterhout}. Most files accessed are small, but size distributions are skewed enough that most data transferred is from large files. Writes are less common than reads, and most files are short-lived. Files access is typically sequential, and most files are read from or written to in their entirety. These trends have proved resiliant over time, though the scale has increased and largest files in typical systems have grown much larger \cite{ruemmler,gibson98b}.

As cache sizes grow, more requests can be satisfied without consulting the disk, and designs can assume that many read requests can be served from memory. All writes whose effects are not quickly undone have to go to disk eventually, so despite being fewer overall, writes requests can dominate the mix of operations that penetrates the cache and reaches the disk. Furthermore, updates often involve metadata changes as well, potentially requiring multiple costly disk seeks even for small updates.

The log-structured file system (LFS) addressed this problem by borrowing from database design and making the entire file system an append-only log. Writes are gathered and written as sequential chunks on disk, with relevant metadata rewritten instead of updating existing structures directly \cite{rosenblum}. \emph{Journaling} file systems apply the same idea to other file system types, logging only the intent to update metadata. Once the log is committed, the conventional structures can be updated asynchronously while still guaranteeing durability in the face of a system crash \cite{hagmann,sweeney,tweedie}. Breaking the chain of required metadata commits is also the objective of \emph{soft updates}, a technique that involves careful rearrangement of data in the buffer cache to permit delays and reordered writes \cite{ganger94}.

Studies comparing FFS with LFS \cite{seltzer95} and journaling with soft updates \cite{seltzer00} reveal a complicated picture. The requirement of cleaner process in LFS to reclaim space from the log interferes under some workloads, and updates to clustered file systems reduce the ordered-write problem enough to keep it competitive with small file updates. Will all of these systems, it is careful attention to the motion of the disk head that leads to good performance.

% TODO: This is a really weak section
What's the point?  How does this apply to Envoy?

In distributed systems or even local RAID arrays \cite{patterson}, higher-order effects can further distort the picture \cite{stein05}.

\subsection{Client-server file systems}

If local file systems can be characterized by how they manage disk head motion, distributed file systems are dominated by concerns of data placement and cache management. For a single host, cache management is easy. The OS has a monopoly on the disk and can reconcile any concurrent requests directly. Complications are mainly concerned with deciding when to commit writes to disk to ensure durability in the face of a system crash. Distributed systems must also consider consistency between caches on multiple machines. If a cache delivers an out-of-date version of a file, the application may be led to produce incorrect results.

The Sun Network File System (NFS) was the first widely-used file system for sharing files across hosts \cite{sandberg}. NFS serves many clients from a single server, which hosts the persistent and canonical version of a file. Cache policy is unspecified, with no explicit support from the server. Clients generally cache reads and writes in memory and check with the server before relying on old cache entries (typically in the range of 10s of seconds). Thus an update made by one client is only detected by another when the first has sent the update to the server and the second has checked for an update. Clients can send a constant stream of \texttt{stat} requests to check for updates, but they cannot hasten a delayed update from another client, so they can never be assured of having the latest version of a file. An update to the protocol helped reduce traffic somewhat by performing implicit \texttt{stat} requests and including the results with common operations \cite{pawlowski,callaghan}, but the fundamental problem remained as clients still delay writes to the server. The latest update, NFSv4, includes features to improve cache management by \emph{delegating} complete control of individual files to clients and \emph{revoking} the delegation when other clients seek concurrent access.

Other client-server systems addressed the problem in different ways. The Andrew File System (AFS) \cite{satyanarayanan85,howard} uses the client's disk as a persistent cache with close-to-open semantics. In this scheme, the cache is validated at file open time, and changes forwarded to the server at file close time. To reduce validity checks at file open time, a client can be given a \emph{callback}, meaning that it can assume the file is current unless explicitly notified by the server. These changes improved scalability by enlarging the effective cache size on clients and reducing the load on the server, but they still left open the possibility of conflicting updates by multiple clients. Coda extended AFS to explore the problem of conflict resolution, opening up access to mobile users and allowing clients to continue operating when disconnected from the network \cite{satyanarayanan90,mummert}. DEcorum went the other way, extending AFS to strengthen cache consistency using tokens, as well as interopperate better with existing file systems and reduce recovery time after a crash \cite{kazar}.

The Sprite team observed that---despite the popularity of such systems---most applications do not explicitly account for inconsistencies introduced by the file system, so loosening consistency guarantees in favor of performance gains is a dangerous tradeoff. They sacrificed some performance for full cache consistency by disabling caching for files under contention \cite{baker91,nelson,welch}. Since concurrent access is relatively rare \cite{kistler}, this did not pose a significant problem for overall performance.

In all distributed file systems that permit sharing, there must either be a canonical version of the file (or block \cite{mcgregor}) or some way to reconcile conflicting updates \cite{kistler}. In the former case, some participant is usually nominated as the owner of a particular file through a lease \cite{gray89}, token passing \cite{burrows,mann,kazar}, or some other scheme. Any other host wanting access to the latest version must coordinate through the owner. Ownership may go to the host that provides storage, the one actively using the file, or a third manager host that connects the two \cite{blaze,keleher}. In Envoy, ownership goes to an active user, which then acts as a synchronous server to other users. Unlike Sprite, the principle user can continue to cache the file locally and share its cache with other concurrent users.

Another possibility is to disallow write sharing. The Cedar file system \cite{schroeder} makes all shared files immutable, and turns the problem into one of versioning \cite{gifford}. Venti takes this a step further by storing all file versions permanently and addressing them by a hash of their contents \cite{quinlan}. Since files are never changed or deleted, the store collects a complete history of all historical states of the file system. In workstation environments, it is possible for storage capacity to grow faster than storage is consumed, making this a feasible system, or past versions can be selectively removed as in the local file system Elephant \cite{santry}.

A less drastic approach is to permit snapshots \cite{hitz}, then mark historic versions as read-only with no requirement for cache coordination \cite{warfield}. Envoy employs this dichotomy between active and read-only file versions, implementing cache management only for mutable objects. It does not coalesce identical read-only objects like Venti or Farsite \cite{douceur02}. It could be extended to do so using a lazy, asynchronous process, but it reduces the need by promoting file system forking with copy-on-write as a management tool to avoid creating many of the duplicates in the first place.

Where snapshot systems typically use copy-on-write to transparently combine old data with new, some systems allow explicit stacking of file systems. Spring \cite{khalidi} allows file systems to be layered with optionally synchronized updates as a mechanism for extending functionality by layering in new features. Plan~9 \cite{pike90} allows any file system mount to be layered over another and their contents combined to give each user a custom view of local and remote file systems \cite{pike92}. The copy-on-write NFS server I developed for the XenoServers project \cite{kotsovinos04b} allowes layering instructions to be put in files, where the server immediately notices them and reconfigures the user's view of the file system. These systems can be used as a way to fork a file system, by sharing the common base image and capturing changes in a private layer. Over time this can lead to complex hierarchies of layers, and such systems rely on the semantics of their backing file systems. Envoy can be used in conjunction with a stacking layer, but it already provides explicit support for snapshots and file system forks.

\subsection{Serverless file systems}

While creative caching can alleviate the problem somewhat, all client-server architectures have inherent scaling problems. As a single point of contact for all clients, a server's load grows in proportion to the number of clients, and it also represents a single point of failure. In addition, the duties of a server tend to make it unsuitable for other uses, so such an architecture calls for a dedicated server. As workstations have grown in power, harnessing their excess capacity to cooperate on large problems has become increasingly attractive \cite{anderson95a}.

In xFS, the traditional roles of a server are split and distributed to the clients to yield a serverless architecture \cite{anderson95b}. Hosts can act as clients, managers that coordinate data placement, and/or storage servers that provide disk space, similar to the file system of the earlier LOCUS distributed operating system \cite{walker}. Cache coherency is achieved through an explicit consistency protocol where conflicting client requests are detected and managed.

In addition, the xFS team observed that modern networks make retrieval from a peer's cache faster than from a disk \cite{dahlin94b}, so sharing and coordinating cache across hosts can yield benefits over discrete local caches \cite{dahlin94a}. The resulting protocol was so complex that the team had to employ a formal protocol verifier to get a working implementation \cite{wang98}. While a valid way to manage complexity, reducing complexity through design may be preferable, especially in storage systems where correctness is paramount.

Farsite also targets a workstation environment, but assumes that participating machines are not trusted \cite{adya}. This requires encrypting data and using complex Byzantine agreement protocols instead of trusting hosts that have been assigned management roles. It also calls for a higher replication factor to guard against malicious attacks as well as hardware failures \cite{dahlin94a}, and makes cache sharing between hosts less practical. These restrictions are imposed by the environment, again highlighting the importance of matching storage design to the expected conditions.

The downside of having workstations double as servers is that the server function is not completely isolated from the other activities of the workstation. Runtime activity may be less predictable than on a dedicated server. Server load is normally determined by the aggregate activity of many clients instead of that of a single user, and a user may also switch a workstation off without notice. While a server can also fail unexpectedly, administration generally focuses on making this an infrequent event. The same trends that lead to excess capacity in workstations make dedicated servers cheap and powerful, without the additional complexity of a heterogeneous management environment. Despite numerous studies \cite{bolosky,douceur99,douceur01} showing feasibility and systems developed and tested \cite{adya,walker}, no serverless file system has seen widespread use for general-purpose computing.

While Envoy has similar goals for serving a location-independent file hierarchy to many untrusted clients, putting it in a trusted cluster environment changes the assumptions significantly. Hardware virtualization allows malicious clients to coexist on hardware with trusted server processes, and the complexity of Byzantine failure models can be avoided. Managed hardware also means that replication factors can be planned around hardware failure rates and load balancing without worrying about hosts switching off frequently.

\subsection{Wide-area file systems}

Carrying the idea of distributed file systems to the logical extreme leads to global file systems, running on hosts throughout the world and serving millions of clients located anywhere. Latency and available bandwidth become major concerns in this environment, and the inability to trust hosts forces widely-distributed file systems---as was true with Farsite in a smaller setting---to focus mainly on managing replicas for availability and reliability.

\subsubsection{Systems with servers}

Ficus \cite{guy} and Echo \cite{birrell93} link servers together to form a single, global file hierarchy, with transparent navigation between the discrete volumes that make up the system. In Echo, entire volumes are replicated in tightly-synchronized groups with one \emph{primary} server and one or more \emph{standby} servers. A token-passing scheme allows clients to cache files locally but still maintain global consistency \cite{mann}. Ficus relaxes the synchonization requirements in favor of optimistic concurrency, where conflicts must be resolved after being detected. Volume replicas are loosely synchronized, and may hold copies of only a subset of the files logically contained in the volume. Because of the loose synchronization, updates can be made to any file that has at least one replica available \cite{popek}. This permits continued operation in the face of network partitions or other failures, similar to the use of the persistent cache in Coda \cite{kistler}.

An early version of xFS \cite{wang93} also follows a two-tier model, where clusters of nearby hosts work together and share a single \emph{consistency server}. The consistency server acts as a proxy for the group when communicating in the wide area, and as a coordinator for operations within the cluster. Requests are served from the cache of a local host when possible, and forwarded to a remote cluster when necessary. Consistency is maintained through tokens that permit local caching for multiple readers or a single writer. To reduce the state that must be tracked, tokens cover entire groups of files.

JetFile \cite{gronvall} and Pangaea \cite{saito02a} rely on pervasive replication with little overall structure. JetFile  uses specialized servers for a few metadata functions, but most operations happen directly between clients. In both systems clients maintain replicas of all files they are interested in, and optimistic concurrency control permits disconnected operation. They differ in how updates are propogated. Pangaea maintains a replica graph for each file and pushes updates to other clients \cite{saito02b}, while JetFile makes extensive use of IP multicast to locate replicas and announce changes, and clients pull updates when required. When conflicting updates are detected in either system, the versions must be reconciled explicitly, generally using a last-writer-wins policy.

% Fluid Replication \cite{kim} 2002
% Segank: distributed mobile storage system \cite{sobti} 2004
% Optimizing for bandwidth consumpion \cite{muthitacharoen}
% Prospero \cite{neuman}

\subsubsection{Peer-to-peer systems}

Relying on trusted servers restricts the audience for a wide-area file system to large organizations and service providers who can maintain widely-dispursed networks. Peer-to-peer systems use resources volunteered by participants on large numbers of machines.

The simplest systems are read-only file distribution schemes. Bittorrent tracks all clients with a central manager, but data blocks are transferred mainly between clients \cite{cohen,pouwelse}, which request blocks they have not yet received from peers that have already downloaded them. Avalanche \cite{gkantsidis} uses network coding to decrease the incidence of ``rare'' blocks that make it difficult for clients to complete the last steps of a file download. Both systems feature capacity that grows with the number of users, without consuming excessive bandwidth at the server. Such systems are mainly useful for sharing the cost of distributing static content with those who benefit from it, and not for general-purpose storage needs.

Distributed hash tables (DHTs) such as CAN \cite{ratnasamy}, Pastry \cite{rowstron01b}, Chord \cite{stoica}, and Tapestry \cite{zhao01} provide distributed lookup services on peer-to-peer networks, and have been used as the basis for wide-area file systems. CFS \cite{dabek} and PAST \cite{rowstron01a} implement read-only systems using content-based addressing, similar to Venti \cite{quinlan}, but using an underlying DHT to locate object replicas. Other systems implement mutable file systems over similar substrates, storing data blocks, files, or content-based fragments \cite{rabin81} as immutable objects distributed across the network. Pasta \cite{moreton} stores metadata in special index blocks, each of which is associated with an asymmetric cryptographic key. The key is used to locate the index block (instead of using a hash of the contents as for normal data blocks) and to protect changes to it, allowing the index to change but retain a unique, static ID. Eliot \cite{stein02} stores metadata outside the immutable substrate, creating a seperate, writable system that references the read-only data indexed by the DHT. Ivy \cite{muthitacharoen02} stores all user changes to the user's log, which is then made available to others through the DHT, and each user consults as many logs as necessary to construct a coherent view of the file system.

Peer-to-peer file systems all suffer from the transience of users. Volunteered resources can be withdrawn without notice, and high levels of replication are required to ensure accessibility and availability \cite{blake,rabin89}. Latency is also high in the wide area, so further replication and caching is necessary to make performance acceptable. OceanStore \cite{kubiatowicz} was designed as a global network of smaller, highly-connected \emph{pools} that cooperate closely, with additional clustering of files that are regularly accessed together. The prototype, called Pond \cite{rhea}, uses Tapestry to organize virtual resources (data and managers), but it also forms localized \emph{rings} of participants to manage Byzantine agreement without the high latency typical of DHTs. The common element of these approaches is that they use tiered systems to reclaim some of the benefits of locality while still providing a global service.

Envoy has some structural parallels with these systems. While latency between machines in a cluster is much lower, Envoy still caches data on disk and in memory near the client. Locality is pursued at the machine level to aggregate the storage requirements of a set of virtual machines, with data ultimately replicated and spread throughout the cluster as storage objects.

The relationship of the present work to global storage systems is more complicated than a passing architectural resemblence, however. Envoy and the service clusters that host it are designed for commercial providers that locate computational resources near storage and fast network access. Instead of trying to hide the distance between users and data, service clusters share the goal of XenoServers \cite{reed} to move computation to a resource-rich environment. As a basic platform, service clusters backed by Envoy storage form the building blocks of global service networks, including storage services.

\subsection{Clusters}

The path from local file systems to wide-area file systems is generally progressive, with concerns about cache management, trust, latency, replica placement, and reliability growing at each step. Clusters have a curious mixture of characteristics compared to the other systems described here, and the demands put on the storage system reflect this. They resemble the workstation groups that use serverless file systems, but the group of host is trusted and centrally managed more like an individual machine. They can scale to thousands or tens of thousands of nodes, but are connected by high-speed networks with redundant components.

Because of their high component count, clusters experience node failure regularly. 

feeble attempts:
Layered clustering in Cuckoo, aggregate NFS servers \cite{klosterman} 2002
Mirage aggregates multiple NFS servers into a single, virtual server \cite{baker02}.

\subsubsection{Storage layers}

AutoRAID \cite{wilkes95} (appliace)
Zebra \cite{hartman93} is network RAID \cite{patterson}
Swarm \cite{hartman99} generic logged striped storage.
Petal \cite{lee95,lee96}
Frangipani \cite{thekkath}
Ursa Minor \cite{abd-el-malek} 2005

\subsubsection{Object systems}

Network-attached secure disks (NASD) \cite{gibson97,gibson98a}
Network-attached storage (NetAPP) \cite{hitz}
file systems for SANS \cite{burns}
highly concurrent storage controllers \cite{amiri} 2000
Bandwidth can become a problem as scale grows \cite{hospodor}, requiring careful architecture choices.

\subsubsection{Other systems}

Lustre \cite{lustre}
Google file system \cite{ghemawat}
Self-* Storage \cite{ganger03}
FAB \cite{frolund,saito04,ji}
Ceph \cite{weil} 2006

\subsubsection{Virtual machines}

CoWNFS \cite{kotsovinos04b}
GRID VMs \cite{zhao04}
Parallax \cite{warfield}
Ventana \cite{pfaff}

\section{Summary}
